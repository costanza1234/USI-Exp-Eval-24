\documentclass[unicode,11pt,a4paper,oneside,numbers=endperiod,openany]{scrartcl}

\renewcommand{\thesubsection}{\arabic{subsection}}

\input{assignment.sty}
\begin{document}


\setassignment
\setduedate{Monday, 11 November 2024, 11:59 PM}

\serieheader{Experimentation and Evaluation}{2024}{\textbf{Students:} Davide Frova, Costanza Rodriguez Gavazzi}{}{Project 1}{}
\newline


\subsection{Abstract}

Short (120-130 words) summary of your entire report. Give the reader a quick idea of what you did and what the main findings were (if you prepare this report ahead of time, leave out the findings until after you finish the analysis).


\subsection{Introduction}

Introduce the topic of investigation to the reader and motivate why you did the experiment. Note that in our case, writing “because I was told to by the course instructor” is not a valid answer. Please assume that you are trying to answer a certain relevant question and motivate its relevance. (In a “real” study report, you would need to also discuss any relevant prior research results here. Given our setting, however, we skip any “related work” consideration.) Your final paragraph of the introduction should outline your proposed experiment.


    \subsubsection{Hypotheses}

    Write down your (falsifiable!) hypotheses here. Each hypothesis must include independent and your dependent variables. You must write down your hypotheses before you do your experiment!

\subsection{Method}

In the following subsections, describe everything that a reader would need to replicate your experiment in all important details.


    \subsubsection{Variables}

    Explicitly identify the independent variable(s) (i.e., what you as the experimenter manipulate): 
    [Independent variable : Levels]\\
    
    Explicitly identify the dependent variable(s) (i.e., what you measure):
    [Dependent variable : Measurement Scale]\\

    Explicitly identify any important control variable(s) (i.e., what you keep constant): Note that you do not need to spell out items that you do not expect to make a significant difference! E.g., do not list room temperature unless you believe that minor differences have an impact! Only list variables here that you think are important to keep at a certain level.
    [Control variable : Fixed Value]\\


    \subsubsection{Design}

    Check off the characteristics of your experimental design:\\
    Type of Study (check one):\\
    Observational Study,\\
    Quasi-Experiment,\\
    Experiment\\
    \hfill
    Number of Factors (check one):\\
    Single-Factor Design,\\
    Multi-Factor Design,\\
    Other\\
    \hfill
    Explain, (1) in text using terminology from the book and lectures and (2) with a figure (similar to those used in Chapter 3 of the Field and Hole book), what kind of experiment you did.


    \subsubsection{Apparatus and Materials}

    Describe in sufficient detail any relevant “props” that you used in your experiment. This could be the computer you used (exact model and specification), the software used (URL, version numbers), the way you measured, e.g., time (A stopwatch? A background process on the computer that got automatically triggered?). Omit needless detail (e.g., think whether details like the size of the table the laptop was placed on, or the hard disk size, might have affected your results or not).


    \subsubsection{Procedure}
    Describe how you used your props and/or the participants to perform your actual experiment, i.e., how you actually carried out a single experimental run. What was done to the participants? What did they have to do? How long did each session take (unless this is an actual dependent variable)? If you did not have participants, explain, e.g., what software was started by whom in what order.



\subsection{Results}

    \subsubsection{Visual Overview}

    Provide an insightful overview of the data you collected. This requires some engineering from your part, to find a good degree of summarization: On one end of the spectrum, you don't summarize, and report hundreds of raw measurement values in a block of text. On the other end of the spectrum, you report a single number (like a mean value). Both approaches are bad.\\

    Instead, use appropriate visual summaries (such as scatter plots, histograms, box plots, or empirical cumulative distribution functions) to show the distribution of your data. If you have a very small number of measurement values, then report all of them in a well organized table (where rows and/or columns correspond to different levels of different factors)


    \subsubsection{Descriptive Statistics}

    For each group or condition, summarize the set of measured values with a "five-number summary": minimum, first quartile, median, third quartile, and maximum.\\
    
    Make sure you explain in your words what these statistics mean “in plain English”, but don't yet interpret them (this is for the Discussion section).


\subsection{Discussion}

    \subsubsection{Compare Hypotheses with Results}

    Provide a brief restatement of the main results from the previous section, and if (or if not) these support your research hypothesis.\\

    If there is a discrepancy between your hypothesis and the results of your experiment, speculate about why you were unable to find evidence to support your hypothesis. 


    \subsubsection{Limitations and Threats to Validity}

    Acknowledge any faults or limitations your study has, and how seriously these affect your results. How could these be remedied in future work?


    \subsubsection{Conclusions}
    End with the main conclusions that can be drawn from your study.

\subsection{Appendix}
    \subsubsection{Materials}

    Any documents you used for your informed consent (information sheets, consent) or as part of your apparatus (e.g., manual, hand-out), please include them here.


    \subsubsection{Reproduction Package}

    Before, during, and after the experiment you collected all kinds of data. Don't ever throw such data away! Any plots, tables, summaries, and statistics provided in this report should be recreatable from the raw data you have.\\

    If you only collected a small amount of data, put it in this Appendix right here.\\

    If you collected data in forms that are better kept in separate files, then zip up those files, and submit them as a "reproduction package" supporting this report.

\end{document}
